% !TEX encoding = UTF-8
% !TEX TS-program = pdflatex
% !TEX root = ../tesi.tex

%**************************************************************
\chapter{Introduzione}
\label{cap:introduzione}
%**************************************************************

Introduzione al contesto applicativo.\\

\noindent Esempio di utilizzo di un termine nel glossario \\
\gls{api}. \\

\noindent Esempio di citazione in linea \\
\cite{site:agile-manifesto}. \\

\noindent Esempio di citazione nel pie' di pagina \\
citazione\footcite{womak:lean-thinking} \\

%**************************************************************
\section{L'azienda}

Infocert S.p.A. è una delle più importanti Certification Authority a livello europeo
e fornisce servizi di firma digitale, Posta Elettronica Certificata (PEC)[g], Sistema
Pubblico di Identità Digitale (SPID)[g], e fatturazione elettronica. L’obiettivo principale
dell’azienda è quello di rendere disponibili tutti gli strumenti per la creazione di un
ufficio digitale, fornendo ai propri clienti soluzioni paperless. Queste vengono fornite
ad altre aziende e a professionisti.

%**************************************************************
\section{L'idea}

Il progetto di stage è nato da un bisongno dell'azienda: analizzare l'utilizzo di un framework nuovo per sviluppare applicazioni moderne ed esplorare i principi della Chaos Engineering per una eventualmente graduale integrazione ne ciclo di sviluppo aziendale con lo scopo di creare prodotti di maggior qualità e ridurre il costo di mantenimento dek software.
In particolare il framework coinvolto è Akka ed è stato deciso di realizzare una nuova versione di un prodotto aziendale già esistente.
Infine è stato deciso di eseguire dei test per analizzare le performance tra l'applicativo appena sviluppato e la sua versione precedente.

%**************************************************************
\section{Organizzazione del testo}

\begin{description}
    \item[{\hyperref[cap:processi-metodologie]{Il secondo capitolo}}] descrive ...
    
    \item[{\hyperref[cap:descrizione-stage]{Il terzo capitolo}}] approfondisce ...
    
    \item[{\hyperref[cap:analisi-requisiti]{Il quarto capitolo}}] approfondisce ...
    
    \item[{\hyperref[cap:progettazione-codifica]{Il quinto capitolo}}] approfondisce ...
    
    \item[{\hyperref[cap:verifica-validazione]{Il sesto capitolo}}] approfondisce ...
    
    \item[{\hyperref[cap:conclusioni]{Nel settimo capitolo}}] descrive ...
\end{description}

Riguardo la stesura del testo, relativamente al documento sono state adottate le seguenti convenzioni tipografiche:
\begin{itemize}
	\item gli acronimi, le abbreviazioni e i termini ambigui o di uso non comune menzionati vengono definiti nel glossario, situato alla fine del presente documento;
	\item per la prima occorrenza dei termini riportati nel glossario viene utilizzata la seguente nomenclatura: \emph{parola}\glsfirstoccur;
	\item i termini in lingua straniera o facenti parti del gergo tecnico sono evidenziati con il carattere \emph{corsivo}.
\end{itemize}