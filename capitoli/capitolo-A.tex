% !TEX encoding = UTF-8
% !TEX TS-program = pdflatex
% !TEX root = ../tesi.tex

%**************************************************************
\chapter{Conclusione}
%**************************************************************

\section{Ulteriori passi nell'adozione della \textit{Chaos Engineering}: esperimenti e test}
Dopo un primo periodo di tempo in cui si effettuano esperimenti esplorativi sul sistema si incamerano certamente un certo numero di esperimenti che hanno avuto successo e che hanno contribuito ad aumentare le fiducia nel prodotto.
Questi esperimenti però non dovrebbero diventare semplicemente storia ma contribuire ancora alla fiducia nel prodotto diventando parte integrante del processo di continous integration.
Ogni esperimento che ha successo e del quale sono state risolte tutte le eventuali criticità individuate dovrebbe entrare in questo processo con lo scopo di verificare, a cadenza fissa o ad ogni build del prodotto, se sono state introdotte delle regressioni.
Il passo successivo quindi agli esperimenti esplorativi che abbiamo effettuato durante lo \textit{stage} sarebbe l'integrazione dei risultati positivi ottenuti nel processo di continous integration di MICO2.
Il processo aziendale che porta ad una completà maturità nella \textit{Chaos Engineering} passa necessariamente attraverso degli esperimenti con un \textit{blast radius} molto piccolo, per poi incrementarlo e infine giungere ad integrare tutti gli esperimenti riusciti o risolti come test della Continous Integration per evitare di introdurre regressioni.


\section{Chaos Maturity Model}
L'adozione della \textit{Chaos Engineering} è un processo lungo che può durare anche anni e si compone di alcuni livelli fondamentali descritti nel Chaos Maturity Model, un modello che ci fornisce un modo per tenere traccia del livello di \textit{Chaos Engineering} all'interno di un'organizzazione.
Le due metriche su cui si basa il CMM sono la sofisticazione e l'adozione della \textit{Chaos Engineering}, esse indicano rispettivamente quanto la \textit{Chaos Engineering} sia radicata nelle pratiche aziendali e quanto le applicazioni pratiche di quest'ultima siano sofisticate e automatizzate all'interno dell'azienda.
Per ottenere i risultati migliori dalla Chaos Enginering c'è bisogno di un buon livello in entrambe le metriche.
Senza sofisticazione gli esperimenti sono rischiosi e poco affidabili, senza adozione invece avremo strumenti estremamente potenti ma che non hanno alcun effetto positivo sul nostro prodotto.

\subsection{Sofisticazione}
La sofisticazione incrementa la validità e la sicurezza degli esperimenti e può variare tra i team e tra i progetti a seconda degli strumenti e tecniche utilizzate.
Possiamo dividere la sofisticazione in quattro livelli:
\begin{description}
    \item[Elementare]: gli esperimenti non si svolgono in produzione, sono eseguiti manualmente e non vengono utilizzate metriche di business
    \item[Semplice]: gli esperimenti simulano il traffico di produzione, sono eseguiti automaticamente ma necessitano di monitoraggio manuale, i risultati vengono documentati e storicizzati, si incomincia ad aggiungere la latenza nella rete tra gli eventi disponibili
    \item[Sofisticata]: gli esperimenti vengono eseguiti in produzione, tutte le fasi dell'esperimento sono automatizzate, gli esperimenti sono integrati con la continous delivery, i \textit{tool} permettono di tracciare i risultati nel tempo e il confronto interattivo
    \item[Avanzata] gli esperimenti vengono eseguiti in ogni ambiente, sia sviluppo che produzione, anche il design e la preparazione dell'esperimento sono automatizzati, i \textit{tool} consentono di fare previsioni automatizzate sulle capacità del \textit{software} e perdite nel fatturato dal risultato degli esperimenti 
\end{description}

\subsection{Adozione}
L'adozione misura quanto si estende l'ambito degli esperimenti di \textit{Chaos Engineering} e anche questa metrica si divide in quattro livelli:
\begin{description}
    \item[Nell'ombra]: pochi sistemi coperti dagli esperimenti, non c'è coscienza della \textit{Chaos Engineering} come pratica aziendale, gli esperimenti vengono effettuati con scarsa frequenza
    \item[Investimento]: gli esperimenti sono standardizzati in azienda, la pratica della \textit{Chaos Engineering} è assegnata part-time, più di un team è coinvolto in questa pratica
    \item[Adozione]: Un team è dedicato solamente alla \textit{Chaos Engineering}, tutti i servizi critici sono coperti da esperimenti regolari, vengono organizzati in maniera occasionale dei "game days"
    \item[Cultura]: tutti i servizi vengono coperti da esperimenti frequenti, la \textit{Chaos Engineering} è parte del processo di sviluppo e la partecipazione ad eventi riguardanti quest'ultima sono pratica comune nell'azienda   
\end{description}

\subsection{Grafico}
Per visualizzare meglio la situazione dei team e degli strumenti secondo queste metriche possiamo disegnare un grafico ponendo le due metriche sulle rispettive assi cartesiane:
\begin{figure}[H]
    \centering
    \includegraphics[width=14cm]{chen_1501.png}
    \label{tab:grafico-vuoto-cmm}
    \caption{Grafico vuoto del Chaos Maturity Model}
\end{figure}
Poi possiamo collocare team e strumenti oppure l'intera azienda sul grafico, in questo modo avremo un'indicazione di ciò che abbiamo già fatto e ciò che invece può essere migliorato.
In questo grafico ad esempio vediamo come siano stati collocati i principali strumenti di Netflix in base al grado di adozione e sofisticazione che offrono.
\begin{figure}[H]
    \centering
    \includegraphics[width=14cm]{chen_0901.png}
    \label{tab:grafico-vuoto-cmm}
    \caption{Grafico riempito del Chaos Maturity Model}
\end{figure}

\section{Consuntivo}
Rispetto alla pianificazione originaria il consuntivo è terminato in positivo, il progetto infatti è stato terminato con un giorno di anticipo rispetto a quanto preventivato, quindi in 312 ore rispetto alle 320 preventivate.

\section{Obiettivi raggiunti}
Durante lo stage sono stati completati tutti gli obiettivi richiesti e quelli desiderabili: Lo studio dei principi della \textit{Chaos Engineering} è andato molto nel dettaglio e mi ha permesso di redarre anche una lista di best practices sull'applicazione della \textit{Chaos Engineering} e che è lasciata disponibile all'azienda, ho analizzato diversi strumenti nei loro vantaggi e svantaggi e valutato quali fossero i migliori per gli scopi dello stage.
Insieme al team di sviluppo ho progettato l'applicazione MICO2 guardando ai rischi della \textit{Chaos Engineering} e documentato questo processo, sempre insieme al team inoltre ho esplorato il sistema che abbiamo sviluppato con degli esperimenti per aumentare la nostra fiducia del sistema stesso, per finire abbiamo redatto un confronto tra la vecchia versione monolitica MICO e la nuova versione reactive in Kubernetes.
Durante tutto lo stage inoltre sono stato introdotto al framework Scrum e alla gestione dei progetti secondo le regole aziendali.

\section{Retrospettiva peronale sul progetto di \textit{stage}}
Complessivamente il progetto di \textit{stage} è stata una bella sfida, l'argomento ha suscitato da subito il mio interesse anche se la mancanza di veri e propri pilastri nella disciplina mi ha costretto a reperire le informazioni da fonti diverse.
La parte di studio è stata sicuramente la più interessante perchè non sapevo nulla di questo argomento e ha da subito catturato il mio interesse.
Lo \textit{stage} ha permesso di conoscere come con metodo scientifico e collaborazione si possa migliorare la qualità di \textit{software} troppo complessi per essere sempre prevedibili.

Anche gli esperimenti di Chaos Engineering sono stati interessanti e soprattutto molto coinvolgenti, soprattutto perchè mi ha permesso di collaborare con tutto il team di sviluppo.
Gli esperimenti però hannno anche dimostrato come la teoria sia estremamente interessante e promettente ma il cammino per adottare questa disciplina sia lungo e consista anche nell'abbracciare nuovi rischi.

\subsection{ChaosToolkit}
Lo strumento principale di cui ci siamo avvalsi per gli esperimenti di \textit{Chaos Engineering} si è rivelato un'ottima scelta, direi quasi indispensabile per standardizzare la struttura degli esperimenti.
Nonostante abbia richiesto uno sforzo aggiuntivo per configurare lo strumento e le sue espansioni proprio questa possibilità di configurare su misura gli strumenti di cui si ha bisogno lo rende uno strumento versatile e molto potente.


