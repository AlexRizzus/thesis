% !TEX encoding = UTF-8
% !TEX TS-program = pdflatex
% !TEX root = ../tesi.tex

%**************************************************************
\chapter{Conclusione}
%**************************************************************

\section{Ulteriori passi nell'adozione della Chaos Engineering: esperimenti e test}
Dopo un primo periodo di tempo in cui si effettuano esperimenti esplorativi sul sistema si incamerano un certo numero di esperimenti che hanno avuto successo e che hanno contribuito ad aumentare le fiducia nel prodotto.
Questi esperimenti però non dovrebbero diventare semplicemente storia ma contribuire ancora alla fiducia nel prodotto diventando parte integrante del processo di continous integration.
Ogni esperimento che ha successo e del quale sono state risolte tutte le eventuali criticità individuate dovrebbe entrare in questo processo con lo scopo di verificare, a cadenza fissa o ad ogni build del prodotto, se sono state introdotte delle regressioni.
Il processo aziendale che porta ad una completà maturità nella Chaos Engineering passa necessariamente attraverso degli esperimenti con un blast radius molto piccolo, per poi incrementarlo e infine giungere ad integrare tutti gli esperimenti riusciti o risolti come test della Continous Integration per evitare di introdurre regressioni.


\section{Chaos Maturity Model}
L'adozione della Chaos Engineering è un processo lungo che può durare anche anni e si compone di alcuni livelli fondamentali descritti nel Chaos Maturity Model, un modello che ci fornisce un modo per tenere traccia del livello di Chaos Engineering all'interno di un'organizzazione.
Le due metriche su cui la CMM sono la sofisticazione e l'adozione, senza sofisticazione gli esperimenti sono rischiosi e poco affidabili, senza adozione invece avremo strumenti estremamente potenti ma che non hanno alcun effetto sul nostro prodotto.

\subsection{Sofisticazione}
La sofisticazione incrementa la validità e la sicurezza degli esperimenti e può variare tra i team e tra i progetti a seconda degli strumenti e tecniche utilizzate.
Possiamo dividere la sofisticazione in quattro livelli:
\begin{description}
    \item[Elementare]: gli esperimenti non si svolgono in produzione, sono eseguiti manualmente e non vengono utilizzate metriche di business
    \item[Semplice]: gli esperimenti simulano il traffico di produzione, sono eseguiti automaticamente ma necesssitano di monitoraggio manuale, i risultati vengono documentati storicizzati, si incomincia ad aggiungere la latenza nella rete tra gli eventi disponibili
    \item[Sofisticata]: gli esperimenti vengono eseguiti in produzione, tutte le fasi dell'esperimento sono automatizzate, gli esperimenti sono integrati con la continous delivery, i tool permettono di tracciare i risultati nel tempo e il confronto interattivo
    \item[Avanzata] gli esperimenti vengono eseguiti in ogni ambiente, sia sviluppo che produzione, anche il design e la preparazione dell'esperimento sono automatizzati, i tool consentono di fare previsioni automatizzate sulle capacità del software e perdite nel fatturato dal risultato degli esperimenti 
\end{description}

\subsection{Adozione}
L'adozione misura quanto si estende l'ambito degli esperimenti di Chaos Engineering e anche questa metrica si divide in quattro livelli:
\begin{description}
    \item[Nell'ombra]: pochi sistemi coperti dagli esperimenti, non c'è coscienza della Chaos Engineering come pratica aziendale, gli esperimenti vengono effettuati con scarsa frequenza
    \item[Investimento]: gli esperimenti sono standardizzati in azienda, la pratica della Chaos Engineering è assegnata part-time, più di un team è coinvolto in questa pratica
    \item[Adozione]: Un team è dedicato solamente alla Chaos Engineering, tutti i servizi critici sono coperti da esperimenti regolari, vengono organizzati in maniera occasionale dei "game days"
    \item[Cultura]: tutti i servizi vengono coperti da esperimenti frequenti, la Chaos Engineering è parte del processo di sviluppo e la partecipazione ad eventi riguardanti quest'ultima sono pratica comune nell'azienda   
\end{description}

\subsection{Grafico}
Per visualizzare meglio la situazione dei team e degli strumenti in uno secondo queste metriche possiamo disegnare un grafico ponendo le due metriche sulle rispettive assi cartesiane:
\begin{figure}[h]
    \centering
    \includegraphics[width=14cm]{chen_1501.png}
    \caption{Grafico vuoto del CMM}
    \label{tab:grafico-vuoto-cmm} 
\end{figure}
Poi possiamo collocare team e strumenti oppure l'intera azienda sul grafico, in questo modo avremo un'indicazione di ciò che abbiamo già fatto e ciò che invece può essere migliorato.
In questo grafico ad esempio vediamo come siano stati collocati i principali strumenti di Netflix in base al grado di adozione e sofisticazione che offrono.
\begin{figure}[h]
    \centering
    \includegraphics[width=14cm]{chen_0901.png}
    \caption{Grafico vuoto del CMM}
    \label{tab:grafico-vuoto-cmm}
\end{figure}

\section{Retrospettiva sul progetto di stage}
Complessivamente il progetto di stage è stata una bella sfida, l'argomento ha suscitato da subito il mio interesse anche se la mancanza di veri e propri pilastri nella disciplina mi ha costretto a reperire le informazioni da fonti diverse.
Mi ha permesso di conoscere come con metodo scientifico e collaborazione si possa migliorare la qualità di software troppo complessi per essere sempre prevedibili.

La parte pratica invece ha dimostrato come la teoria sia estremamente interessante e promettente ma il cammino per adottare questa disciplina sia lungo e consista in parte nel lasciarsi alle spalle vecchie abitudini.


\section{Consuntivo}
Rispetto alla pianificazione originaria il consuntivo è terminato in positivo, il progetto infatti è stato terminato con un giorno di anticipo rispetto a quanto preventivato.



\epigraph{Citazione}{Autore della citazione}



