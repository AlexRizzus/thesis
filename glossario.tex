
%**************************************************************
% Acronimi
%**************************************************************
\renewcommand{\acronymname}{Acronimi e abbreviazioni}

\newacronym[description={\glslink{apig}{Application Program Interface}}]
    {api}{API}{Application Program Interface}

\newacronym[description={\glslink{umlg}{Unified Modeling Language}}]
    {uml}{UML}{Unified Modeling Language}

%**************************************************************
% Glossario
%**************************************************************
%\renewcommand{\glossaryname}{Glossario}

\newglossaryentry{apig}
{
    name=\glslink{api}{API},
    text=Application Program Interface,
    sort=api,
    description={in informatica con il termine \emph{Application Programming Interface API} (ing. interfaccia di programmazione di un'applicazione) si indica ogni insieme di procedure disponibili al programmatore, di solito raggruppate a formare un set di strumenti specifici per l'espletamento di un determinato compito all'interno di un certo programma. La finalità è ottenere un'astrazione, di solito tra l'hardware e il programmatore o tra software a basso e quello ad alto livello semplificando così il lavoro di programmazione}
}

\newglossaryentry{umlg}
{
    name=\glslink{uml}{UML},
    text=UML,
    sort=uml,
    description={in ingegneria del software \emph{UML, Unified Modeling Language} (ing. linguaggio di modellazione unificato) è un linguaggio di modellazione e specifica basato sul paradigma object-oriented. L'\emph{UML} svolge un'importantissima funzione di ``lingua franca'' nella comunità della progettazione e programmazione a oggetti. Gran parte della letteratura di settore usa tale linguaggio per descrivere soluzioni analitiche e progettuali in modo sintetico e comprensibile a un vasto pubblico}
}

\newglossaryentry{blast radiusg}
{
    name=\glslink{blast radiusg}{Blast radius},
    text=Blast radius,
    sort=blast radius,
    description={nella Chaos Engineering con blast radius si intende il danno potenziale che un esperimento è in grado di causare al sistema nello scenario peggiore, è molto importante tenerlo in considerazione nella preparazione di un esperimento poichè un esperimento con un blast radius troppo grande potrebbe causare un grave downtime del sistema}
}

\newglossaryentry{spinnakerg}
{
    name=\glslink{spinnakerg}{Spinnaker},
    text=Spinnaker,
    sort=spinnaker,
    description={è una piattaforma software per la continues delivery sviluppata da Netflix e ora gestita in collaborazione con Google}
}

\newglossaryentry{jsong}
{
    name=\glslink{jsong}{json},
    text=json,
    sort=json,
    description={Javascript Object Notation è un formato per lo scambio di dati fra applicazioni client/server ed è basato sul linguaggio Javascript}
}

\newglossaryentry{pipg}
{
    name=\glslink{pipg}{pip},
    text=pip,
    sort=pip,
    description={pip è un sistema di gestione per i pacchetti scritti in Python}
}

\newglossaryentry{replicasg}
{
    name=\glslink{replicasg}{Replica},
    text=replicas,
    sort=replicas,
    description={un replica è un singolo elemento di un ReplicaSet, ossia un insieme di pod contenenti la stessa applicazione e raggruppati logicamente}
}

\newglossaryentry{go-wrkg}
{
    name=\glslink{go-wrkg}{go-wrk},
    text=go-wrk,
    sort=go-wrk,
    description={go-wrk è uno strumento open-source scritto nel linguaggio Go che permette di effettuare stress-test http per un indirizzo tramite l'utilizzo di diverse goroutines per mandare più richieste simultaneamente}
}

\newglossaryentry{goroutinesg}
{
    name=\glslink{goroutinesg}{goroutine},
    text=goroutines,
    sort=goroutines,
    description={è un thread di esecuzione molto leggero e performante utilizzabile nel linguaggio di programmazione Go}
}

\newglossaryentry{stress-ngg}
{
    name=\glslink{stress-ngg}{stress-ng},
    text=stress-ng,
    sort=stress-ng,
    description={è uno strumento che permette di testare un sistema nei suoi sottosistemi fisici in diversi modi, originariamente era stato pensato per individuare problemi hardware}
}

\newglossaryentry{flowg}
{
    name=\glslink{flowg}{Flow},
    text=flow,
    sort=flow,
    description={Un flow è un insieme di \gls{stepg}, è associato ad un prodotto di Infocert e rappresenta una sequenza di azioni che può considerarsi conclusa quando tutti gli step sono nello stato: completato}
}

\newglossaryentry{stepg}
{
    name=\glslink{stepg}{Step},
    text=step,
    sort=step,
    description={Costituisce un elemento di un flow, può trovarsi in diversi stati tra cui: non completato, in corso, completato}
}